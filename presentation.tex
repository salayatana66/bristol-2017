\documentclass[compress]{beamer}
% --------------------------------------------------------------------------
% Common packages
% --------------------------------------------------------------------------
\usepackage[T1]{fontenc}
\usepackage{graphicx}
\usepackage{multicol}
\usepackage{amsmath}
\usepackage{color}
\usepackage{amsfonts}
\usepackage{amssymb}
\usepackage[mathscr]{euscript}
%%%%%%%%%%%%%%%%%%%%%%%%%%%%%%%%%%%%%%%%%%%%%%%%%%%%%%%%%%%%%%%%%%%%%%%%%%


% Erweiterte Tabellenfunktionen
\usepackage{tabularx,ragged2e}
\usepackage{booktabs}
% for the transition diagram
\usepackage{tikz}
\usetikzlibrary{arrows}
\setbeameroption{show notes}
% Listingserweiterung
\usepackage{listings}
\lstset{ %
  language=[LaTeX]TeX,
  basicstyle=\normalsize\ttfamily,
  keywordstyle=,
  numbers=left,
  numberstyle=\tiny\ttfamily,
  stepnumber=1,
  showspaces=false,
  showstringspaces=false,
  showtabs=false,
  breaklines=true,
  frame=tb,
  framerule=0.5pt,
  tabsize=4,
  framexleftmargin=0.5em,
  framexrightmargin=0.5em,
  xleftmargin=0.5em,
  xrightmargin=0.5em
}


% ------------------------------------------------------------------------------
% Tikz
% ------------------------------------------------------------------------------
\usetikzlibrary{arrows,positioning}
\tikzset{
  % Define standard arrow tip
  >=stealth',
  % Define style for boxes
  punkt/.style={
    rectangle,
    rounded corners,
    draw=black, very thick,
    text width=6.5em,
    minimum height=2em,
    text centered},
  % Define arrow style
  pil/.style={
    ->,
    thick,
    shorten <=2pt,
    shorten >=2pt,}
}

% NaturProzess




\newcommand{\captionslide}[1]{
  {\setbeamercolor{background canvas}{bg=hsrmWarmGreyLight}
    \begin{frame}[plain]
      \addtocounter{framenumber}{-1}
      \begin{center}
        \vfill\usebeamerfont{section
          title}\textcolor{black}{\MakeUppercase{#1}}\vfill
      \end{center}
    \end{frame}
  }
}



% --------------------------------------------------------------------------
% Load theme
% --------------------------------------------------------------------------
\usetheme[nosectionpages]{hsrm}


\usepackage{dtklogos} % must be loaded after theme
\usepackage{tikz}
%%%%%%%%%%%%%%%%%%%%%%%%%%%%%%%%%%%%%%%%%%%%%%%%%%%%%%%%%%%%%%%%%%%%%%%%%%
%% Math Macros
%%%%%%%%%%%%%%%%%%%%%%%%%%%%%%%%%%%%%%%%%%%%%%%%%%%%%%%%%%%%%%%%%%%%%%%%%%
\theoremstyle{plain} %to change the headings as suggested by Kleiner
\newtheorem{lem}[equation]{Lemma}
\newtheorem{prop}[equation]{Proposition}
\newtheorem{thm}[equation]{Theorem}
\newtheorem{cor}[equation]{Corollary}
\theoremstyle{definition}
\newtheorem{defn}[equation]{Definition}
\newtheorem{ques}[equation]{Question}
\theoremstyle{remark}
\newtheorem{exa}[equation]{Example}
\def\ccgroup{\mathbb{G}}%Carnot group
\def\hmeas#1.{\mathscr{H}^{\setbox0=\hbox{$#1\unskip$}\ifdim\wd0=0pt 1
    \else #1\fi}} %Hausdorff measure with default argument by
\def\real{{\mathbb{R}}}
\def\zahlen{{\mathbb Z}}
\def\Xint#1{\mathchoice
   {\XXint\displaystyle\textstyle{#1}}%
   {\XXint\textstyle\scriptstyle{#1}}%
   {\XXint\scriptstyle\scriptscriptstyle{#1}}%
   {\XXint\scriptscriptstyle\scriptscriptstyle{#1}}%
   \!\int}
\def\XXint#1#2#3{{\setbox0=\hbox{$#1{#2#3}{\int}$}
     \vcenter{\hbox{$#2#3$}}\kern-.5\wd0}}
\def\ddashint{\Xint=}
\def\av{\Xint-}
\DeclareMathOperator\diam{diam}%diametre
\def\biglip{\text{\rm Lip}} %local lipschitz constant
\def\glip#1.{{\bf L}(#1)} %global Lipschitz constant
\def\smllip{\text{\rm lip}} %minimal variation
%%%%%%%%%%%%%%%%%%%%%%%%%%%%%%%%%%%%%%%%%%%%%%%%%%%%%%%%%%%%%%%%%%%%%%%%%%
\title{Calculus on Metric Spaces: Beyond the Poincar\'e Inequality}
\date{}
\author{Andrea Schioppa \\ Data Science, \\ Booking.com BV}
\subtitle{New Examples of Differentiability Spaces}

\begin{document}
\maketitle
\addtocounter{framenumber}{-1}
\note{
  \textbf{Abstract:} \\
  We discuss a framework introduced by J.~Cheeger (1999) to
  differentiate Lipschitz maps defined on metric measure spaces which
  admit Poincar\'e inequalities, and discuss (the first) examples on which it is
  still possible to differentiate despite the infinitesimal geometry being
  incompatible with the Poincar\'e inequality.
}
%%%%%%%%%%%%%%%%%%%%%%%%%%%%%%%%%%%%%%%%%%%%%%%%%%%%%%%%%%%%%%%%%%%%%%%%%%
\section{Differentiability spaces}
\captionslide{Differentiability spaces: why we care?}
%%%%%%%%%%%%%%%%%%%%%%%%%%%%%%%%%%%%%%%%%%%%%%%%%%%%%%%%%%%%%%%%%%%%%%%%%%
\begin{frame}[fragile]\frametitle{Geometric Group Theory}
  \begin{itemize}
  \item Mostow (1968), ``Quasi-conformal mappings in $n$-space and
  the rigidity of the hyperbolic space forms''.
  \item Pansu (1989), ``Carnot-Caratheodory Metrics and
    quasi-Isometries of rank-one symmetric spaces''
  \item Pansu's Rademacher Theorem: Any Lipschitz or
    quasi-Conformal map $f:\ccgroup_1\to\ccgroup_2$ between Carnot
    groups is differentiable at $\hmeas\dim\ccgroup_1.$-a.e.~point. The
    derivative is a group homomorphism commuting with dilations. 
  \end{itemize}
\end{frame}
%%%%%%%%%%%%%%%%%%%%%%%%%%%%%%%%%%%%%%%%%%%%%%%%%%%%%%%%%%%%%%%%%%%%%%%%%%
\begin{frame}[fragile]\frametitle{The abstract Poincar\'e inequality (1)}
  \begin{itemize}
  \item Heinonen \& Koskela (1998), ``Quasiconformal maps in metric
    spaces with controlled geometry''
  \item The key ingredients are \textbf{doubling measures} and
    \textbf{the Poincar\'e inequality} (PI-spaces)
  \item Rigidity for Fuchsian buildings, Sobolev spaces, $p$-harmonic
    maps, \dots
  \end{itemize}
\end{frame}
%%%%%%%%%%%%%%%%%%%%%%%%%%%%%%%%%%%%%%%%%%%%%%%%%%%%%%%%%%%%%%%%%%%%%%%%%%
\begin{frame}[fragile]\frametitle{The abstract Poincar\'e inequality
    (2)}
  \begin{itemize}
  \item  The classical $(1,p)$-Poincar\'e inequality:
    \begin{equation}
        \av_{B}|u-u_B|\,d\hmeas N.
        \le C \diam(B)\,\left(
        \av_B\|\nabla u\|^p\,d\hmeas N.
      \right)^{1/p}.
    \end{equation}
  \item In the metric setting: $\hmeas N.\to\mu$,
    $\|\nabla u\|\to$ a ``surrogate'' $g$ (\textbf{upper gradient}).
  \item If $u$ is Lipschitz can take:
      \begin{equation}
    \biglip u(x)=\limsup_{y\ne x\to x}\frac{|u(y)-u(x)|}{d(x,y)}.
  \end{equation}
  \end{itemize}
\end{frame}
%%%%%%%%%%%%%%%%%%%%%%%%%%%%%%%%%%%%%%%%%%%%%%%%%%%%%%%%%%%%%%%%%%%%%%%%%%
\begin{frame}[fragile]\frametitle{A finite-dimensionality argument}
  \begin{itemize}
  \item Cheeger (1999): If $\mu$ is doubling, at $\mu$-a.e.~$f$ the
    blow-ups of $f$ are $p$-harmonic and the gradient has constant
    modulus \textbf{(generalized linearity)}.
  \item Cheeger (1999): In a PI-space a linear subspace of generalized-linear
    functions is finite-dimensional.
  \item Cheeger's Rademacher: Fix Lipschitz coordinate
     functions $\{\psi_i\}_{i=1}^N$; any $f$ is differentiable
     $\mu$~a.e.:
         \begin{equation}
    \label{eq:alt_rad_ch}
      f(x')=f(x)+\sum_{i=1}^{N}a_i(x)\left(\psi_{i}(x')-\psi_{i}(x)\right)+o\left(d_X(x,x')\right). 
    \end{equation}

  \end{itemize}
\end{frame}
%%%%%%%%%%%%%%%%%%%%%%%%%%%%%%%%%%%%%%%%%%%%%%%%%%%%%%%%%%%%%%%%%%%%%%%%%%
\begin{frame}[fragile]\frametitle{Applications}
  \begin{itemize}
  \item Legitimates first-order calculus for PI-spaces (e.g.~now
    $\biglip f$ is really $\|df\|$).
  \item Provides non-embeddability results $f:X\to\real^n$.
  \item The idea of differentiation has been generalized to other
    targets: $f:X\to L^\infty$ (\textbf{metric differentiation})) and
    $f:X\to L^1$ (\textbf{the cut-metric representation} and a
    counterexample of Cheeger \& Kleiner to the Goemans-Linial conjecture).
  \end{itemize}
\end{frame}
%%%%%%%%%%%%%%%%%%%%%%%%%%%%%%%%%%%%%%%%%%%%%%%%%%%%%%%%%%%%%%%%%%%%%%%%%% 
\section{The derivative along curves}
\captionslide{Constructing the derivative along curves: Alberti
  representations}
%%%%%%%%%%%%%%%%%%%%%%%%%%%%%%%%%%%%%%%%%%%%%%%%%%%%%%%%%%%%%%%%%%%%%%%%%%
\begin{frame}[fragile]\frametitle{Differentiability spaces}
  \begin{itemize}
  \item A metric measure space $(X,\mu)$ such that any Lipschitz
    $f:X\to\real$ is differentiable $\mu$-a.e.~is a
    \textbf{differentiability space}.
  \item Keith 2004, ``A differentiabile structure for metric measure
    spaces'': replace the Poincar\'e inequality with the ``Lip-lip''
    inequality: $\mu$-a.e.:
    \begin{equation}
      \biglip f \le K \smllip f,
    \end{equation}
    where
    \begin{equation}
      \smllip f(x) = \liminf_{r\to0}\sup_{y\in B(x,r)}\frac{|f(x)-f(y)|}{r}.
    \end{equation}
  \end{itemize}
\end{frame}
%%%%%%%%%%%%%%%%%%%%%%%%%%%%%%%%%%%%%%%%%%%%%%%%%%%%%%%%%%%%%%%%%%%%%%%%%%
\begin{frame}[fragile]\frametitle{Drawbacks}
  \begin{itemize}
  \item Both Cheeger \& Keith's proof are non-constructive. No curves.
  \item Cheeger proved better properties for PI-spaces (e.g.~$\biglip
    f = \smllip f$, stability of PI under blow-up).
  \item No example of Keith's spaces not countable unions of
    positive-measure subsets of PI-spaces.
  \end{itemize}
\end{frame}
%%%%%%%%%%%%%%%%%%%%%%%%%%%%%%%%%%%%%%%%%%%%%%%%%%%%%%%%%%%%%%%%%%%%%%%%%%
\begin{frame}[fragile]\frametitle{The PI-rectifiability
    ``conjecture''}
  \begin{itemize}
  \item \textbf{Strong form}, Cheeger-Kleiner-S.~(2016): Any differentiability
    space is a countable union of positive-measure subsets of
    PI-spaces.
  \item \textbf{Weak form}: A.e.~the blow-ups/tangents of a differentiability
    space are PI-spaces.
  \item Held belief to be true (had a paper rejected as the ``expert''
    strongly believed in it),
    the PI-inequality should be ``necessary'' to have calculus.
  \end{itemize}
\end{frame}
%%%%%%%%%%%%%%%%%%%%%%%%%%%%%%%%%%%%%%%%%%%%%%%%%%%%%%%%%%%%%%%%%%%%%%%%%%
\begin{frame}[fragile]\frametitle{The Radon-Nikodym property}
  \begin{itemize}
  \item An \textbf{RNP-Banach space} $B$:
    any Lipschitz $f:\real\to B$ is differentiable $\hmeas 1.$-a.e; $c_0$,
    $l^p$ ($1\le p<\infty$), $L^p([0,1])$ ($1<p<\infty$)
  \item Cheeger \& Kleiner (2009): ``Differentiability of Lipschitz
    maps from metric
    measure spaces to Banach spaces with the
    Radon–Nikodym property'': if $(X,\mu)$ is PI, $f:X\to B$ is
    differentiabile
  \item Study the derivative along curves; Preiss' observation: $\mu$
    must admit an Alberti representation.
  \end{itemize}
\end{frame}
%%%%%%%%%%%%%%%%%%%%%%%%%%%%%%%%%%%%%%%%%%%%%%%%%%%%%%%%%%%%%%%%%%%%%%%%%%
\begin{frame}[fragile]\frametitle{Alberti representations}
  \begin{itemize}
  \item An Alberti representation is a Fubini-like decomposition of
    $\mu$:
    \begin{equation}
      \mu = \int_{\Gamma}g\hmeas 1._\gamma\,dP(\gamma).
    \end{equation}
  \item Bate (2014), ``Structure of measures in Lipschitz
    differentiability spaces'': characterizes differentiability spaces
    using Alberti representations
  \item S.~(2015--2016), shows that the ``Lip-lip'' inequality
    is an \textbf{equality}, \textbf{necessary and sufficient} to have
    differentiability, and that a.e.~tangents of differentiability spaces are
    differentiability spaces.
  \item Cheeger-Kleiner-S.: ``Infinitesimal structure of
    differentiability spaces, and metric differentiation''.
  \end{itemize}
\end{frame}
%%%%%%%%%%%%%%%%%%%%%%%%%%%%%%%%%%%%%%%%%%%%%%%%%%%%%%%%%%%%%%%%%%%%%%%%%%
\begin{frame}[fragile]\frametitle{Fragmented connectivity}
  \begin{itemize}
  \item To prove a PI-inequality it is necessary to construct families of
    curves joining points.
  \item Bate \& Li (2016), ``The geometry of Radon-Nikodym Lipschitz
    differentiability spaces'': if all Lipschitz $f:X\to \bigoplus_{l^1}l_n^\infty$ are
    differentiable, can connect points avoiding sets of low density.
  \item Eriksson-Bique (2016), ``Classifying Poincaré inequalities and
    the local
    geometry of RNP-differentiability spaces'': RNP-differentiability
    spaces are PI-rectifiable.
  \end{itemize}
\end{frame}
%%%%%%%%%%%%%%%%%%%%%%%%%%%%%%%%%%%%%%%%%%%%%%%%%%%%%%%%%%%%%%%%%%%%%%%%%% 
\section{New examples}
\captionslide{New Examples: an alternative approach to quantitative
  differentiation}
%%%%%%%%%%%%%%%%%%%%%%%%%%%%%%%%%%%%%%%%%%%%%%%%%%%%%%%%%%%%%%%%%%%%%%%%%%
\begin{frame}[fragile]\frametitle{A PI-unrectifiable example}
  \begin{itemize}
  \item S.~(2016): An example of a differentiability space which is
    PI-unrectifiable.
  \item $X$ has a disconnected tangent at a.e.
  \item Any Lipschitz $f:X\to l^2$ is differentiable a.e.
  \item For $\varepsilon>0$ there is an
    a.e.~\textbf{non}-differentiable $f:X\to l^{3+\varepsilon}$.
  \item In the metric setting differentiability depends on the target!
  \item Can ``probe'' Banach spaces with differentiability spaces.
  \end{itemize}
\end{frame}
%%%%%%%%%%%%%%%%%%%%%%%%%%%%%%%%%%%%%%%%%%%%%%%%%%%%%%%%%%%%%%%%%%%%%%%%%%
\begin{frame}[fragile]\frametitle{Building blocks}
  \begin{itemize}
  \item Non-quasiconvex diamonds.
  \item Non-selfsimilar Inverse limit system (compare Laakso, Cheeger
    \& Kleiner)
  \item ``Tritanopic'' chromatic labels, Alberti representations and
    the horizonatl gradient.
  \end{itemize}
\end{frame}
%%%%%%%%%%%%%%%%%%%%%%%%%%%%%%%%%%%%%%%%%%%%%%%%%%%%%%%%%%%%%%%%%%%%%%%%%%
\begin{frame}[fragile]\frametitle{Disconnected tangents}
  \begin{itemize}
  \item At a generic point there is a disconnected tangent.
  \item It is possible to join pairs of points by using
    ``horizontal'' paths and, if necessary, by using at most one
    ``jump''
  \item The jumps are unavoidable and break down the classical
    differentiation argument.
  \end{itemize}
\end{frame}
%%%%%%%%%%%%%%%%%%%%%%%%%%%%%%%%%%%%%%%%%%%%%%%%%%%%%%%%%%%%%%%%%%%%%%%%%%
\begin{frame}[fragile]\frametitle{Classical quantitative differentiation}
  \begin{itemize}
  \item Jones (1988), ``Lipschitz and bi-Lipschitz functions''
  \item Quantify how $f:Q\subset \real\to\real$ is close to a ``linear function''
    on dyadic subdivisions $Q_{n,j}$
  \item The error:
    \begin{equation}
      \alpha(f,Q_{n,j}) = \frac{1}{\hmeas 1.(Q_{n,j})}\inf_l\sup_{x\in
        Q_{n,j}} |f(x)-l(x)|
    \end{equation}
  \item A bound on ``bad cubes'':
    \begin{equation}
      \sum_{Q_{n,j}: \alpha(f,Q_{n,j})\ge\varepsilon}\hmeas
      1.(Q_{n,j})\lesssim\log(\varepsilon)\varepsilon^{-2}\hmeas 1.(Q). 
    \end{equation}
  \end{itemize}
\end{frame}
%%%%%%%%%%%%%%%%%%%%%%%%%%%%%%%%%%%%%%%%%%%%%%%%%%%%%%%%%%%%%%%%%%%%%%%%%%
\begin{frame}[fragile]\frametitle{(Tail)-recursive quantitative
    differentiation}
  \begin{itemize}
  \item Take $f:X\to l^2$, we would like to claim that if $Q$ is a
    quasiconvex diamond $1/n$-squeezed, $\|f(c_{green})-f(c_{red})\|_2 =
    O(\varepsilon \diam Q/n)$.
  \item The error is $\varepsilon/n$, decreasing in the tail of the
    decomposition.
  \item The decomposition is no longer dyadic or self-similar.
  \item Using harmonic functions can prove:
    \begin{equation}
      \sum_{
        \text{{\rm $Q$ is $\varepsilon$-bad}}}\frac{\varepsilon^2}{n_Q^3}\mu(Q)\lesssim
      \glip f.^2.
    \end{equation}
  \end{itemize}
\end{frame}
%%%%%%%%%%%%%%%%%%%%%%%%%%%%%%%%%%%%%%%%%%%%%%%%%%%%%%%%%%%%%%%%%%%%%%%%%%
\begin{frame}[fragile]\frametitle{Further directions}
  \begin{itemize}
    \item Classify $p$ : $f:X\to L^p$ is differentiable, $p=3$ is
      critical.
    \item The examples have analytic dimension $3$, lower to $1$.
  \end{itemize}
\end{frame}
\end{document}
